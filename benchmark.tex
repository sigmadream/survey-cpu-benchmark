\documentclass[12pt]{article}

\usepackage{geometry}
\usepackage{amsmath, amssymb, amsthm}
\usepackage{setspace}
\usepackage{graphicx}
\usepackage{enumitem}
\usepackage{kotex}
\usepackage{subcaption}
\usepackage{hhline}
\usepackage{xcolor}
\usepackage{url}
\usepackage{listings}
\usepackage{courier}
\usepackage[htt]{hyphenat}
\usepackage{hyperref}

\lstdefinestyle{customlisp}{
  belowcaptionskip=1\baselineskip,
  breaklines=true,
  frame=L,
  xleftmargin=\parindent,
  language=Lisp,
  showstringspaces=false,
  basicstyle=\footnotesize\ttfamily,
  keywordstyle=\bfseries\color{green!40!black},
  commentstyle=\itshape\color{purple!40!black},
%   identifierstyle=\color{blue},
%   stringstyle=\color{orange},
}

\title{모바일 기반 R5RS Scheme 인터프리터의 성능 비교에 사용될 알고리즘 소개}
\author{한상곤}
\date{December 11, 2023 updated: \today}

\begin{document}

\maketitle

\section{Pi Digits(pidigits)}\label{pi-digitspidigits}
이 알고리즘은 파이의 자릿수를 생성하는 것입니다. 해당 알고리즘은 연산 속도를 비교하기 위해서 순차 알고리즘을 사용하여 계산을 진행합니다. 파이의 자릿수를 구하는 알고리즘은 Unbounded Spigot Algorithms for the Digits of Pi \cite{gibbons2006unbounded}를 참고하였습니다.

\lstinputlisting[caption=Pi Digits 예제, style=customlisp]{01_pidigits.rkt}


\section{FANNKUCH}\label{fannkuch}

판쿠흐 벤치마크(FANNKUCH Benchmark)는 Kenneth R. Anderson과 Duane Rettig의 Performing Lisp Analysis of the FANNKUCH Benchmark\cite{anderson1994performing}에 소개된 알고리즘입니다. $n$이 무한대가 될 때 $n*log(n)$일 것으로 추측됩니다

\lstinputlisting[caption=FANNKUCH 예제, style=customlisp]{02_FANNKUCH.rkt}

\section{Spectral Norm}\label{spectral-norm}

$a_{11}=1$, $a_{12}=1/2$, $a_{21}=1/3$, $a_{13}=1/4$, $a_{22}=1/5$, $a_{31}=1/6$ 등의 항목이 있는 무한 행렬 $A$의 스펙트럼 노름의 값을 구하는 문제로 Hundred-Dollar, Hundred-Digit Challenge Problems의 3번 문제\cite{keithbriggs2002solution}입니다. 구글의 입사문제로 유명하며, 행렬 및 벡터 연산을 비교하는데 주로 사용됩니다.

\lstinputlisting[caption=Spectral Norm 예제, style=customlisp]{03_spectral-norm.rkt}

\section{mandelbrot}\label{mandelbrot}

프랙털의 일종으로, 수열 ${Z_n}$의 절대값이 무한대로 발산하지 않는 복소수 $c$의 집합에 관한 점화식$(Z_0 = 0$, $Z_{n+1} = Z_{n}^2 + C)$에 관한 문제입니다. 관련 알고리즘은 \href{https://en.wikipedia.org/wiki/Mandelbrot_set}{위키피디아}를 참고하였습니다.


\bibliographystyle{unsrt}
\bibliography{benchmark}

\end{document}